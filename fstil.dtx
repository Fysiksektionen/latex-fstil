%\iffalse meta-comment
% !TEX TS-program = XeLaTeX
% !TEX encoding = utf-8
%
%\iffalse meta-comment
% Copyright (C) 2013 THS Fysiksektionen
% Authored by: Tomas Lycken (tlycken) and Emil Ringh (eringh)
% .dtx file based on guide by Joseph Wright at http://www.texdev.net/2009/10/06/a-model-dtx-file/
%\fi
%<*internal>
\iffalse
%</internal>
%<*readme>
#LaTeX-paket för Fysiksektionens grafiska profil

Användarinstruktioner finns i pdf-filen fstil.pdf.

###Typsnitt

TTF-filer för de typsnitt som behövs finns bland Fysiksektionens övriga LaTeX-relaterade saker (d.v.s. här: https://github.com/Fysiksektionen/latex/tree/master/fonts).
%</readme>
%<*internal>
\fi
\def\nameofplainTeX{plain}
\ifx\fmtname\nameofplainTeX\else
  \expandafter\begingroup
\fi
%</internal>

%<*install>
\input docstrip.tex
\keepsilent
\askforoverwritefalse

\preamble
----------------------------------------------------------------
fstil --- THS Fysiksektionens grafiska profil
E-mail: bigmac@f.kth.se
Released under the LaTeX Project Public License v1.3c or later
See http://www.latex-project.org/lppl.txt
----------------------------------------------------------------

\endpreamble


\usedir{tex/latex/fstil}
\generate{
  \file{\jobname.sty}{\from{\jobname.dtx}{package}}
}
%</install>
%<install>\endbatchfile

%<*internal>
\usedir{source/latex/fstil}
\generate{
  \file{\jobname.ins}{\from{\jobname.dtx}{install}}
}
\nopreamble\nopostamble
\usedir{doc/latex/fstil}
\generate{
  \file{README.md}{\from{\jobname.dtx}{readme}}
}
\ifx\fmtname\nameofplainTeX
  \expandafter\endbatchfile
\else
  \expandafter\endgroup
\fi
%</internal>

%<*driver>
\documentclass{ltxdoc}
\usepackage{\jobname}
\usepackage[numbered]{hypdoc}
\EnableCrossrefs
\CodelineIndex
\RecordChanges
\begin{document}
\OnlyDescription
  \DocInput{\jobname.dtx}
\end{document}
%</driver>

%
%\StopEventually{^^A
%  \PrintChanges
%  \PrintIndex
%}
%
%<*package>
\NeedsTeXFormat{LaTeX2e}
\ProvidesPackage{fstil}[2013/01/26 v2.0 THS Fysiksektionens grafiska profil]
%</package>
%
%\fi
%\GetFileInfo{\jobname.sty}
%
%\changes{v2.0}{2013/01/25}{Första .dtx-versionen av paketet}
%
%\title{^^A
%  \textsf{fstil} --- THS Fysiksektionens grafiska profil\thanks{^^A
%    Denna fil dokumenterar version \fileversion, senast ändrad \filedate.^^A
%  }^^A
%}
%\author{^^A
%  Tomas Lycken \\ tlycken@kth.se \and Emil Ringh \\ eringh@kth.se^^A
%}
%\date{Publicerad \filedate}
%
%\maketitle
%
%\section{Paketet fstil}
%Paketet \textsf{fstil} är utvecklat för att ge Fysiksektionen en grafisk profil.
%Det ställer om fonten till \textit{Garamond}, ger ett sidhuvud som innehåller sektionens
%logotyp samt en sektionsrelaterad sidfot. Den grafisk profilen är sådan att den
%efterliknar THS grafiska profil, fast anpassad till Fysiksektionen. Detta dokument är 
%skrivet med hjälp av docstrip, och ger ett exempel på hur det kan se ut när man använder
%paketet \textsf{fstil}.
%
%\section{Paket som används}
%
%\textsf{fstil} använder sig av funktionalitet från följande paket:
%\begin{description}
%\item[\textsf{fontenc}, med option \textsf{T1}] Använder den modernare teckenkodningen T1, vilken t.ex. underlättar för specialtecken som åäö.
%    \begin{macrocode}
\RequirePackage[T1]{fontenc}
%    \end{macrocode}
%\end{description}
%
%
%
%
%\Finale